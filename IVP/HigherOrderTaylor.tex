\chapter{Higher Order Methods}
\section{Higher order Taylor Methods}
The Taylor expansion
\[ y(t_{i+1})=y(t_i)+h y^{'}(t_i)+\frac{h^2}{2}y^{'}(t_i)+..+\frac{h^{n+1}}{(n+1)!}y^{n+1}(\xi_i) \]
can be used to design more accurate higher order methods. 
By differentiating the original \addtoindex{Ordinary Differential Equation} $y^{'}=f(t,y)$  higher ordered can be derived 
method it requires the function to be continuous and differentiable.\\
In the general case of Taylor of order n:
\[ w_0=\alpha \]
\[ w_{i+1} = w_i + hT^n(t_i,w_i), \mbox{   for   } \ i=0,...,N-1, \]
where
\begin{equation} T^{n}(t_i,w_i) = f(t_i,w_i)+\frac{h}{2}f'(t_i,w_i)+...\frac{h^{n-1}}{n!}f^{n-1}(t_i,w_i). \end{equation}
\begin{example}
Applying the general Taylor method to create methods of order two and four to
the initial value problem
\[y^{'}=y-t^2+1, \ \ \ 0\leq t \leq 2, \ \ \ y(0)=0.5, \]
from this we have 
\[ f^{'}(t,y(t)) = \frac{d}{dt}(y-t^2+1) = y'-2t=y-t^2+1-2t, \]
\[ f^{''}(t,y(t)) = y-t^2-2t-1, \]
\[ f^{'''}(t,y(t)) = y-t^2-2t-1. \]
From these derivatives we have
\begin{eqnarray*}
T^{2}(t_i,w_i)&=&f(t_i,w_i)+\frac{h}{2}f^{'}(t_i,w_i)\\
&=&w_i-t_i^2+1+\frac{h}{2}(w_i-t_i^2-2t_i+1)\\
&=&\left(1+\frac{h}{2}\right)(w_i-t_i^2-2t_i+1)-ht_i
\end{eqnarray*}
and
\begin{eqnarray*}
T^{4}(t_i,w_i)&=&f(t_i,w_i)+\frac{h}{2}f^{'}(t_i,w_i)\\
& &+\frac{h^2}{6}f^{''}(t_i,w_i)+\frac{h^3}{24}f^{'''}(t_i,w_i)\\
&=&\left(1+\frac{h}{2}+\frac{h^2}{6}+\frac{h^3}{24}\right)(w_i-t_i^2)\\
& &-\left(1+\frac{h}{3}+\frac{h^2}{12}\right)ht_i\\
& & +1+\frac{h}{2}-\frac{h^2}{6}-\frac{h^3}{24}
\end{eqnarray*}
From these equations we have,
Taylor of order two
\[w_0=0.5\]
\[w_{i+1}=w_i+h\left[\left(1+\frac{h}{2}\right)(w_i-t_i^2-2t_i+1)-ht_i
\right] \]
and Taylor of order 4
\begin{eqnarray*}w_{i+1}&=&w_i+h\left[\left(1+\frac{h}{2}+\frac{h^2}{6}+\frac{h^3}{24}\right)(w_i-t_i^2)\right.\\
& &\left.-\left(1+\frac{h}{3}+\frac{h^2}{12}\right)ht_i
 +1+\frac{h}{2}-\frac{h^2}{6}-\frac{h^3}{24}
\right] \end{eqnarray*}
The local truncation error for the 2nd order method is 
\[\tau_{i+1}(h) = \frac{y_{i+1}-y_i}{h} -T^2(t_i,y_i) = \frac{h^2}{6}f^2(\xi_i,y(x_i))\]
where $\xi \in (t_i,t_{i+1})$.
\end{example}
In general if $y \in C^{n+1}[a,b]$
\[\tau_{i+1}(h)=\frac{h^n}{(n+1)!}f^{n}(\xi_i,y(\xi_i))~O(h^n). \]
The issue is that for every differential equation a new method has be to derived.

\newpage
\section{Problem Sheet 2 - Higher Order Methods (Taylor)}
\begin{enumerate}

\item
Apply 2nd Order Taylor Method to approximate the solution of the given initial value problems using the indicated number of time steps. Compare the approximate solution with the given exact solution, and compare the actual error with the theoretical local and global error
\begin{enumerate}
\item
$y'=t-y, \ \ (0\leq t \leq 4)$\\
with the initial condition $y(0)=1,$\\
$N=4$, 
$y(t)=2e^{-t}+t-1,$\\

The Lipschitz constant is determined on  $D=\{(t,y);0\leq t \leq 4, y\in \Re \}.$
\item 
$y'=y-t, \ \ (0\leq t \leq 2)$\\
with the initial condition $y(0)=2,$\\
$N=4$, 
$y(t)=e^{t}+t+1$.\\

The Lipschitz constant is determined on  $D=\{(t,y);0\leq t \leq 2, y\in \Re \}.$
\end{enumerate}
\item
Apply 3rd Order Taylor Method to approximate the solution of the given initial value problems using the indicated number of time steps. Compare the approximate solution with the given exact solution, and compare the actual error with the theoretical local and global error
\begin{enumerate}
\item
$y'=t-y, \ \ (0\leq t \leq 4)$\\
with the initial condition $y(0)=1,$\\
$N=4$, 
$y(t)=2e^{-t}+t-1,$\\

The Lipschitz constant is determined on  $D=\{(t,y);0\leq t \leq 4, y\in \Re \}.$
\item 
$y'=y-t, \ \ (0\leq t \leq 2)$\\
with the initial condition $y(0)=2,$\\
$N=4$, 
$y(t)=e^{t}+t+1$.\\

The Lipschitz constant is determined on  $D=\{(t,y);0\leq t \leq 2, y\in \Re \}.$
\end{enumerate}
\item
Apply the Taylor method to approximate the solution of initial value problem
\[ y'=ty+ty^2, \ \ \ (0\leq t \leq 2), \ \ \ y(0)=\frac{1}{2} \]
using $N=4$ steps.
\end{enumerate}
\newpage